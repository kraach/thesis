%%%%%%%%%%%%%%%%%%%%%%%%%%%%%%%%%%%%%%%%%%%%%%%%%%%%%%%%%%%%%%%%%%%%%%%%%%%%%%%%
% conclusion.tex:
%%%%%%%%%%%%%%%%%%%%%%%%%%%%%%%%%%%%%%%%%%%%%%%%%%%%%%%%%%%%%%%%%%%%%%%%%%%%%%%%
\chapter{Conclusion}
\label{conclusion_chapter}
%%%%%%%%%%%%%%%%%%%%%%%%%%%%%%%%%%%%%%%%%%%%%%%%%%%%%%%%%%%%%%%%%%%%%%%%%%%%%%%%

To measure the tiny polarization signal of the \ac{CMB}, \ac{EBEX} employed a kilopixel array of \ac{TES} bolometers. 
In order to realize the benefits of the space-like environment in which \ac{EBEX} flew, I worked with our collaborators at Berkeley to optimize the detectors for the lower, more stable radiative load at float.  
That meant modifying detector parameters such as their normal resistances, transition temperatures, thermal conductances, and time constants. 
I measured these values for each of the flight wafers, as well as several dozen wafers which didn't fly. 
Table~\ref{tab:design_params} summarizes the design and median measured values of the detector parameters for the flight wafers. 
Chapter~\ref{sec:detector_characterization} discusses the measurements. 

The sensitivity of the detectors was quantified by their \ac{NEP}. 
I used the characterization measurements to predict the \ac{NEP}.
In the dark \ac{ETC}, I was able to achieve the predicted \ac{NEP} for the readout electronics and for the detectors in an overbiased state.
Once the detectors were dropped deep into their transitions, however, the measured noise diverged from the prediction.  
For flight, 


%%%%%%%%%%%%%%%%%%%%%%%%%%%%%%%%%%%%%%%%%%%%%%%%%%%%%%%%%%%%%%%%%%%%%%%%%%%%%%%%
