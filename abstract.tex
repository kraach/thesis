%%%%%%%%%%%%%%%%%%%%%%%%%%%%%%%%%%%%%%%%%%%%%%%%%%%%%%%%%%%%%%%%%%%%%%%%%%%%%%%%
% abstract.tex: Abstract
%%%%%%%%%%%%%%%%%%%%%%%%%%%%%%%%%%%%%%%%%%%%%%%%%%%%%%%%%%%%%%%%%%%%%%%%%%%%%%%%

The \ac{CMB} radiation holds a wealth of information about the evolution of the universe. 
In particular, measurement of the polarization pattern of the \ac{CMB} is a direct probe of the physics of inflation. 
The \ac{EBEX} was a balloon-borne telescope designed to search for inflation's signature on the polarization of the \ac{CMB}.  
%EBEX was built to probe the fundamental physics of inflation, which took place at the earliest conceivable time (on the order of 10$^{34}$~seconds) and at the largest of energy scales (on the order of 10$^{15}$-10$^{16}$~GeV).  
To achieve the high receiver sensitivity necessary to measure a small polarization signal, \ac{EBEX}employed a kilopixel array of transition edge sensor bolometers. 
The bolometer design was modified and optimized for the space-like environment at a float altitude of 36~km. 
The detector characterization measurements, performed in test cryostats and in \ac{EBEX} itself, are reported here. 
We measured the bolometer normal resistances, thermal conductances, critical temperatures, optical efficiencies, time constants, and noise equivalent powers. 
We also report on the detector performance, with a particular focus on sensitivity and noise, from the 2013 Antarctic flight.



%%%%%%%%%%%%%%%%%%%%%%%%%%%%%%%%%%%%%%%%%%%%%%%%%%%%%%%%%%%%%%%%%%%%%%%%%%%%%%%%
