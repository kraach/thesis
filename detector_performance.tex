%%%%%%%%%%%%%%%%%%%%%%%%%%%%%%%%%%%%%%%%%%%%%%%%%%%%%%%%%%%%%%%%%%%%%%%%%%%%%%%%
% detector_performance.tex: Chapter on Detector Performance:
%%%%%%%%%%%%%%%%%%%%%%%%%%%%%%%%%%%%%%%%%%%%%%%%%%%%%%%%%%%%%%%%%%%%%%%%%%%%%%%%
\chapter{Detector Performance}
\label{detector_performance_chapter}
%%%%%%%%%%%%%%%%%%%%%%%%%%%%%%%%%%%%%%%%%%%%%%%%%%%%%%%%%%%%%%%%%%%%%%%%%%%%%%%%

%%%%%%%%%%%%%%%%%%%%%%%%%%%%%%%%%%%%%%%%%%%%%%%%%%%%%%%%%%%%%%%%%%%%%%%%%%%%%%%%
% Detector Yield 
%%%%%%%%%%%%%%%%%%%%%%%%%%%%%%%%%%%%%%%%%%%%%%%%%%%%%%%%%%%%%%%%%%%%%%%%%%%%%%%%
\section{Detector Yield}
\label{sec:yield}
%%%%%%%%%%%%%%%%%%%%%%%%%%%%%%%%%%%%%%%%%%%%%%%%%%%%%%%%%%%%%%%%%%%%%%%%%%%%%%%%


You need to say \ac{EBEX} and \ac{SQUID} before the table ...

\begin{table}[ht!]
\begin{center}
\begin{tabular}{lc|c|c|c|c}
 & & \bf150 & \bf250 & \bf410 & \bf Total \\
\hline 1& Total number of bolometers on wafers & 1120 & 560 & 280 & 1960 \\
\hline 2& Able to read out with \ac{EBEX} electronics & 992 & 496 & 254 & 1742 \\
\hline 3& Passed warm electrical \& visual inspections & 908 & 455 & 232 & 1595 \\
\hline 4& Resistor \& dark \ac{SQUID} channels excluded & 861 & 447 & 213 & 1521 \\
\hline 5& Detectors appearing in .8~K network analysis & 805 & 430 & 187 & 1422 \\
\hline 6& Detectors after SQUID failures removed & 773 & 414 & 155 & 1342 \\
\hline 7& Detectors after noise polluters removed & 676 & 371 & 133 & 1180 \\
\hline 8& Detectors with successful flight IV curves & 504 & 342 & 109 & 955 \\
% not including eccosorb/dark 492 & 317 & 92 & \
\hline
\end{tabular}
\end{center}
\caption{The detector yield broken down by observation frequency band, 150, 250, and 410~GHz, as well as the sum. Each row of the table accounts for detector loss.}
\label{yield_table}
\end{table}

\TAB\ref{yield_table} provides an accounting of \ac{EBEX} detector loss as a function of observation frequency band, as well as the total sum for the experiment.
Each silicon wafer, regardless of observation frequency band, had 140 bolometers and one alignment mark. %for the curious/pedantic
The 150 and 250~GHz wafers were coupled to \ac{LC} boards around the perimeter of the focal plane. These edge \ac{LC} boards each had 125 readout channels, 124 of which were connected to bolometers.  
The 410~GHz wafers were in the center of each focal plane and had central \ac{LC} boards in the space just behind the wafer. The central \ac{LC} boards each had 128 readout channels, 127 of which were connected to bolometers. %(because of the alignment mark). 
The first two rows of the table give the total number of bolometers flown and the total number of bolometers the electronics were capable of reading out. 

At room temperature, the wafers were inspected visually and each bolometer was tested for electrical continuity. 
The visual inspection was done under a microscope and, for example, sometimes revealed incomplete etching evidenced by a column of material from the \ac{TES} to the silicon. 
This was noted as a thermal short and such a detector was not electrically biased. 
For the electrical inspection, we measured the resistance by either probing directly across the wafer bond pads or by probing the leads on the \ac{LC} boards. 
The first method was done in the fabrication clean room and the second method was done after the wafer had been shipped, mounted, and wirebonded to its \ac{LC} board. 
The resistance reading was dominated by the room temperature resistance of the niobium leads. 
The electrical inspection could not identify a short across the \ac{TES} because the typical room temperature resistance of the \ac{TES} was a few ohms, much less than the tens of kiloohms of the niobium leads. 
The electrical inspection did, however, identify which \ac{TES} or leads did not make a complete electrical connection, i.e. were open. 
The warm visual and electrical inspection provided an upper limit of the wafer's yield because the open and thermally shorted detectors were guaranteed not to work, see the third row of \TAB\ref{yield_table}. 

For flight, each wafer had two bolometers replaced by 1~ohm resistors for monitoring read out electronic noise up to the \ac{LC} board, one channel at a low bias frequency and the other channel at a high bias frequency. 
Three \ac{SQUID}s were not attached to bolometer combs due to opens in the microstrips. These combs were modified to monitor read out electronic noise up to the \ac{SQUID}. See \TAB\ref{yield_table} row 4. 

Upon cooling the wafer, there were wired detector channels that showed a reasonable room temperature resistance but did not appear in the network analysis, see \TAB\ref{yield_table} row 5. 
Five \ac{SQUID}s failed to operate during flight, see \TAB\ref{yield_table} row 6 for the yield after the \ac{SQUID} failures. 
Once a wafer was characterized in a dark cryostat, the detectors which degraded the noise performance of their entire comb were identified and their wirebonds were removed, see \TAB\ref{yield_table}, row 7. 
%%Occasionally, upon cooling a wafer a second time, additional detectors go missing from the network analysis. (And some detectors re-appear ... presumably due to poor quality wirebond connections.) See \TAB\ref{yield_table}, line "Survived \ac{EBEX} cooldown and plucking." \comred{remove word plucking and remove bad actor, replace with more clear description. no need to mention reappearances since they seldom happen and we only have conjectures as to why?}

After \ac{EBEX} was launched and reached its float altitude, IV curves were performed and the total number of successful curves is reported in \TAB\ref{yield_table}, line 8. 
The losses between row 7 and row 8 were due to detectors being saturated or failing to transition. 

%(though typically they also failed to turn around during the characterization measurements, the loss isn't counted until flight because there was some hope they might work) or the IV curve exhibiting strange behaviour (e.g. a jump in the current reading). \comred{ben showed pv curve, but not iv curve. maybe just call it a pv curve and refer to detector fab section? NO. with reorganization, first mention of iv/pv curve comes after this table ??}
%
%
