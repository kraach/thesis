%%%%%%%%%%%%%%%%%%%%%%%%%%%%%%%%%%%%%%%%%%%%%%%%%%%%%%%%%%%%%%%%%%%%%%%%%%%%%%%%
%detector_characterization.tex: Chapter on ground characterization measurements:
%%%%%%%%%%%%%%%%%%%%%%%%%%%%%%%%%%%%%%%%%%%%%%%%%%%%%%%%%%%%%%%%%%%%%%%%%%%%%%%%
\chapter{Detector Characterization}
\label{charecterization_chapter}
%%%%%%%%%%%%%%%%%%%%%%%%%%%%%%%%%%%%%%%%%%%%%%%%%%%%%%%%%%%%%%%%%%%%%%%%%%%%%%%%


%%%%%%%%%%%%%%%%%%%%%%%%%%%%%%%%%%%%%%%%%%%%%%%%%%%%%%%%%%%%%%%%%%%%%%%%%%%%%%%%
% Parameter Measurements {{{
%%%%%%%%%%%%%%%%%%%%%%%%%%%%%%%%%%%%%%%%%%%%%%%%%%%%%%%%%%%%%%%%%%%%%%%%%%%%%%%%
\section{Detector Parameter Measurements}
\label{parameter_measurements_section}
%%%%%%%%%%%%%%%%%%%%%%%%%%%%%%%%%%%%%%%%%%%%%%%%%%%%%%%%%%%%%%%%%%%%%%%%%%%%%%%%

%TOWARDS NORMAL RESISTANCES

Given the detector design goals outlined in Section~\ref{}, as well as the limited ability to change the detector settings after the payload was launched, it was vital to carefully characterize each detector wafer before considering it for the \ac{EBEX} focal plane. 
%This section provides an overview of the characterization measurements, the sensitivity predictions, and the measured sensitivity of the detectors at float. Additional detail can be found in \cite{aubin_thesis} \cite{MacDermid_thesis} \cite{MacDermid_SPIE2014}.
Of the more than four dozen detector wafers fabricated and characterized, only 14 were able to make the final cut and earn a prized spot in the \ac{EBEX} focal plane for the \ac{EBEX2013} flight. 

For characterization measurements, we mounted the wafer, Section~\ref{}, coupled it to the readout electronics, Section~\ref{}, and cooled it to sub-Kelvin temperatures, where the exact temperature depended on the testbed. 
The three testbeds used were: a dedicated \ac{EBEX} test cryostat at the University of Minnesota, a test cryostat at McGill University, and the \ac{EBEX} cryostat itself, made dark. 

Figure~\ref{} is a network analysis for a single comb of 16 detectors. 
The network analysis swept a voltage across the comb in frequency and measured the current response of the circuit. 
The multiplexed RLC circuit peaked in current at each LC resonant frequency. 
Around 800~mK, when the niobium leads and aluminum wirebonds were superconducting, fitting the peak width provided the normal resistance of the \ac{TES}. 
The current peak is modeled as a Lorentzian, . 
Each peak is fit individually to a Lorentzian (INCLUDE EQUATION) in order to determine each detector's resonant frequency, which is used to provide the bias signal to the TES. 
The network analysis comb is also fit to a model (INCLUDE EQUATION) to get the stray resistance and inductance in series with the comb. 

